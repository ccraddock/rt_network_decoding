%%%%%%%%%%%%%%%%
%% Background %%
%%

\section{Introduction}

Seed based functional connectivity, variance decomposition, and clustering analyses of resting state and task fMRI data have shown that that brain activity can be broken down into a non-overlapping networks that are consistently found across individuals. Early analyses have extracted these networks from longer duration data (5 minutes or more) but shorter-scale analyses are showing quite a bit of temporal dynamics in these networks, which appear to break down into temporal functional modes (TFM). These TFMs represient transient states of network coherence which occur several times during long blocks of brain activity. Although many of the networks can be attributed behaviors from their appearance in task analyses, the meaning of other networks, and the complex interactions between networks is poorly understood. We have developed a real-time fMRI based method for tracking network activity that can be used to develop temprally adabtive experiments for better probing these networks.

Based on preliminary work in functional and effective connectivity, the activity of brain networks can be modeled as a weighted average of activity in the brain areas involved in the network. If the number of brain areas is small compared to the number of observations, then the weights in the average can be solved directly using well established linear algebra techniques. But, in the case of neuroimaging, were the number of brain areas tends to significantly outnumber observations, alternative methods need to be used to solve the equations. There are a variety of different methods in the statistics and machine learning literature for doing this and one method is support vector regression. 

Once the weights in the linear equation are solved for, they can be applied to newly acquired fMRI data to decode a measure of network activity TR-by-TR. 

\subsection{MRI Acquisition}

Data was acquired on a 3 T Siemens Magnetom TIM Trio scanner (Siemens Medical Solutions USA: Malvern PA, USA) using a 12-channel head coil. Anatomic images were acquired at $1 \times 1 \times 1$ mm$^3$ resolution with a 3D T1-weighted magnetization-prepared rapid acquisition gradient-echo (MPRAGE) sequence \cite{Mugler1990} in 192 sagittal partitions each with a $256 \times 256$ field of view (FOV), 2600 ms repetition time (TR), 3.02 ms echo time (TE), 900 ms inversion time (TI), 8$^\circ$ flip angle (FA), and generalized auto-calibrating partially parallel acquisition (GRAPPA) \cite{Griswold2002} acceleration factor of 2 with 32 reference lines. The sMRI data were acquired immediately after a fast localizer scan and preceded the collection of the functional data.

A gradient echo field map sequence was acquired with the following parameters: TR 500 ms, TE1/TE2 2.72 ms/5.18 ms, FA 55$^\circ$, 64 x 64 matrix, with a 220 mm FOV, 30 3.6 mm thick interleaved, oblique slices, and in plane resolution of $3.4 \times 3.4$ mm$^2$. All functional data were collected with a blood oxygenation level dependent (BOLD) contrast weighted gradient-recalled echo-planar-imaging sequence (EPI) that was modified to export images, as they were acquired, to AFNI over a network interface \cite{Cox1995,LaConte2007}. FMRI acquisition consisted of 30 3.6mm thick interleaved, oblique slices with a \%10 slice gap, TR 2000 ms, TE 30 ms, FA 90$^\circ$, $64 \times 64$ matrix, with 220 mm FOV, and in plane resolution of $3.4 \times 3.4$ mm$^2$. Functional MRI scanning included a three volume “mask” scan and a six-minute resting state scan followed by three task scans (described later) whose order was counterbalanced across subjects.

During all scanning, galvanic skin response (GSR), pulse and respiration waveforms were measured using MRI compatible non-invasive physiological monitoring equipment (Biopac Systems, Inc.). Rate and depth of breathing were measured with a pneumatic abdominal circumference belt. Pulse was monitored with a standard infrared pulse oximeter placed on the tip of the index finger of the non-dominant hand. Skin conductance was measured with disposable passive electrodes that were non-magnetic and non-metallic, and collected on the hand. The physiological recordings were synchronized with the imaging data using a timing signal output from the scanner. Visual stimuli were presented to the participants on a projection screen that they could see through a mirror affixed to the head coil. Audio stimuli were presented through headphones using an Avotec Silent-Scan® pneumatic system (Avotec, Inc.: Stuart FL, USA).

\subsection{MRI acquisition order and online Processing}

The real time fMRI neurofeedback experiment utilizes a classifier based approach for extracting DMN activity levels from fMRI data TR-by-TR, similar to \cite{Craddock2012}. The classifier is trained from resting state fMRI data using a time course of DMN activity extracted from the data using spatial regression to a publicly available template derived from a meta-analysis of task and resting state datasets \cite{Smith2009,fmrib_RSNS}. Several stages of online processing are necessary to perform this classifier training, as well as, denoising of fMRI data in real-time. These stages include calculating transforms required to convert the DMN template from MNI space to subject space, creating white matter (WM) and cerebrospinal fluid (CSF) masks for extracting nuisance signals, and training a support vector regression (SVR) model for extracting DMN activity. The MRI session was optimized to collect the data required for these various processing steps, and to perform the processing, while minimizing delays in the experiment.

After acquiring a localizer, the scanning protocol began with the acquisition of a T1 weighted anatomical image used for calculating transforms to MNI space and white matter and CSF masks. Once the image was acquired it was transferred to a DICOM server on the real-time analysis computer (RTAC), which triggered initialization of online processing. The processing script started AFNI in real-time mode, configured it for fMRI acquisition, and began structural image processing. Structural processing included reorienting the structural image to RPI voxel order, skull-stripping using AFNI’s 3dSkullStrip \cite{Cox1996}, resampling the image to isotropic 2-mm voxels (to reduce computational cost of subsequent operations), segmentation into grey matter (GM), white matter (WM), and cerebrospinal fluid (CSF) using FSL’s FAST \cite{Zhang2001}, and normalization into MNI space using FSL’s FLIRT \cite{Jenkinson2002,Jenkinson2001}. WM and CSF probability maps were binarized using a 90\% threshold. The CSF mask was constrained to the lateral ventricles using an ROI from the AAL atlas to avoid overlap with GM.

In parallel with the structural processing a FieldMap was collected but not used in the online processing. Subsequently, a three volume “mask” EPI scan was acquired and transferred to the RTAC. The three images were averaged, reoriented to RPI, and used to create a mask to differentiate brain signal from background (using AFNI’s 3dAutomask \cite{Cox1996}). The mean image was coregistered to the anatomical image using FSL’s boundary based registration (BBR) \cite{Greve2009}. The resulting linear transform was inverted and applied to the WM and CSF masks to bring them into alignment with the fMRI data. Additionally, the transform was combined with the inverted anatomical-to-MNI transform and applied to the canonical map of the DMN (from \cite{Smith2009}).

Next, a 6-minute resting state scan (182 volumes) was collected and used as training data to create the support vector regression (SVR) model. This procedure involved motion correction followed by a nuisance regression procedure to orthogonalize the data to six head motion parameters, mean WM, mean CSF, and global signal \cite{Friston1996,Fox2005,Lund2006}. A SVR model of the DMN was trained using a modified dual regression procedure in which a spatial regression to the unthresholded DMN template was performed to extract a time course of DMN activity. The Z-transformed DMN time course was then used as labels (independent variable), with the preprocessed resting state data as features (dependent variables), for SVR training (C=1.0, $\epsilon = 0.01$) using AFNI’s 3dsvm tool \cite{LaConte2005}. The result was a DMN map tailored to the individual participant based on preexisting expectations about DMN anatomy and function. After SVR training was completed (generally took less than 2 minutes), the MSIT (198 volumes), Moral Dilemma (144 volumes), and Neurofeedback test (412 volumes) scans were run. The order of the task based functional scans was counterbalanced and stratified for age and sex across participants.

\subsection{Functional MRI Tasks}

\subsubsection{Resting state scan}

Participants were instructed to keep their eyes open during the scan and fixate on a white plus (+) sign centered on a black background. They were asked to let their mind wander freely and if they noticed themselves focusing on any particular train of thought, to let their mind wander away from it.
